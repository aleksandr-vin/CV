\section*{Startup Experience}

\subsection*{Spring 2014}

From Spring of 2014 I was a co-founder \& CEO of Evecon Corp. ---
startup,
\href{https://web.archive.org/web/20160304170331/http://evecon.co/}{\url{https://web.archive.org/web/20160304170331/http://evecon.co/}}

We started developing mobile connectivity and indoor navigation
service but later pivoted to development of online service for
eventing market (was \url{attendlist.co}). My co-founder \& I
graduated from the Founder Institute,
\href{https://fi.co}{\url{fi.co}} in summer 2014. While we were
bootstrapping, I was the only one backend developer where I've used
Scala, Go, Python, Twitter Bootstrap, Angular JS for frontend and
backend and Swift/Objective-C/Java for mobile.

\subsection*{Winter 2015}

In January 2015 we released our first iPhone app on App Store:
BizCardApp. You can watch a demonstration video here:
\href{https://youtu.be/TJJCMxzSrDI}{\url{https://youtu.be/TJJCMxzSrDI}}
for our landing page of that time. Its REST API backend and web server
has test stage and production stage servers with automatic deployment
process (based on git post-receive hook). I've used this stack: Nginx
reverse proxy, Supervisord for supervising servers, Gunicorn as WSGI
server, Flask web framework, SQLAlchemy as ORM, Alembic for DB schema
migrations, Jinja2 templating system, WTForms for POST forms and
Twitter Bootstrap for frontend design.

\subsection*{Summer 2015}

During summer of 2015 we made another app for the local Harley
Davidson event, called St. Petersburg Harley Days,
\href{https://www.linkedin.com/company/st-petersburg-harley®-days/about/}{\url{https://www.linkedin.com/company/st-petersburg-harley®-days/about/}}.

You can find it on App Store
\href{https://itunes.apple.com/us/app/harley-days/id998388857?mt=8}{\url{https://itunes.apple.com/us/app/harley-days/id998388857?mt=8}}
and Google Play
\href{https://play.google.com/store/apps/details?id=co.evecon.hdprototype}{\url{https://play.google.com/store/apps/details?id=co.evecon.hdprototype}}.
The main achievement of that project was a very short time to market:
we made it in 1.5 months, working in our spare time and cutting
features' scope for the goal of meeting the deadline. At the end we
promoted the app during event days to collect installations, and
registered only one known crash for iOS 7. The UI and UX design is not
of its best, but the architecture allowed us to be busy updating only the
content for every event up to recent days.

\subsection*{2016}

We released another mobile app for iOS and Android to engage Harley
Owners in Russia and CIS:
\href{https://itunes.apple.com/us/app/hog-plus/id1434462365?mt=8}{\url{https://itunes.apple.com/us/app/hog-plus/id1434462365?mt=8}}.
The app was designed and implemented for H.O.G. Russia \& CIS. Apart
from the app we developed a back-office to manage loyalty programs and
events in the app.

\subsection*{2020}

During Winter of 2020-2021 we developed a live parking detection system,
that was using public street web cameras to collect photos of parking
areas in a city, upload them to S3 buckets, crop photos to mapped zones
on AWS Lamba, detect cars there with YOLOv4
\href{https://github.com/videoparking/darknet}{\url{https://github.com/videoparking/darknet}}
and store results in AWS Timestream. Service was able to say how full
the parking area was at the moment (based on history data) and was coloring
zones on a map. We freezed the project in the Spring of 2021. Public components
can be viewed here
\href{https://github.com/videoparking}{\url{https://github.com/videoparking}}.
