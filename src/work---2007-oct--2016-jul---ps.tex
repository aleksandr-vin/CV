\subsection*{October 2007--July 2016}

\textit{Senior Software Engineer at Peter-Service, CJSC
\href{https://billing.ru}{\url{billing.ru}}, Saint-Petersburg, Russia}

Peter-Service is a big software company (was about 1000 employees
across Russia), it develops and supports billing systems for telco.

\subsubsection*{Last project (mid 2015--mid 2016)}

Full-stack logging infrastructure for distributed architecture. It
solves two problems: (1) aggregating logs from distributed services in
one place with indexed full-text search and (2) runtime controlling of
the produced log detalization level for specific
request-id-s/user-logins/sessions on each node of the distributed
service. The solution consists of log Aggregation infrastructure, log
Analytics service and distributed services Dynamic Log Control
service. Aggregation and Analytics are based on ELK stack by
\href{https://elastic.co}{\url{elastic.co}} ---
Elasticsearch-Logstash-Kibana. Dynamic Log Control service was
designed as pluggable agents for Talend ESB and Tomcat with RESTful
API microservice that synchronize agents via Zookeeper. Allowing
logging on all servers, operations that were originated by specific
user sessions and have that logs being produced in DEBUG level with
minimal effort and overhead. Distributed deployment was implemented in
Ansible.

\begin{itemize}[noitemsep, nosep]
  \item \textbf{Languages:} Java, Ansible
  \item \textbf{Frameworks/Libraries:} Swagger, Logback, Jackson, OSGi
Pax Logging, Apache Curator
  \item \textbf{Services:} Elasticsearch, Zookeeper, Logstash,
Filebeat, Kibana
\end{itemize}

\subsubsection*{Third project (end 2012--mid 2015)}

Distributed high-availability system for providing traffic
rate-limiting/shaping and Level 7 DDoS protection (based on HAProxy,
designed solely from scratch) with dynamic rules in DSL, dynamic
reconfiguration by Zookeeper and a RESTful JSON API. The main design
goal was to develop a scalable solution that can be run on a farm of
low-level machines and when capacities can be scaled on demand. This
goal was achieved mostly by using a "shared-nothing" model and by
implementing performance-critical parts in C language using HAProxy as
a base engine.

HAProxy is a high-performance open source TCP and HTTP load balancer
written completely in pure C. It has a threadless "shared-nothing"
design.  I've done analysis of its sources, designed the architecture
of extension that was needed for our appliance and implemented it with
unit- and performance tests, checking the result with profiler. While
embedding the CLIPS (C Language Integrated Production System) into the
HAProxy, I was using Valgrind to find the memory leak that appeared
under load, finally I've discovered that it was an expected behaviour
of the garbage collector, specifically implemented in CLIPS, and I was
able to workaround the leak.

\begin{itemize}[noitemsep, nosep]
  \item \textbf{Languages:} C, Python, Java, JavaScript, CLIPS
  \item \textbf{Frameworks/Libraries:} Protobuf, ZeroMQ, Python PLY,
Python Mock, CMocka
  \item \textbf{Tools:} Emacs, Strace, GDB, Bash, cURL, JIRA,
Confluence, GIT, Rpmbuild, Valgrind, GNU Autotools, ipset/iptables,
RRDtool, Icinga
  \item \textbf{Services:} Couchbase, Zookeeper, CLIPS
  \item \textbf{Web:} Angular JS, Java EE, Spring, Camel
\end{itemize}

\subsubsection*{Second project (beg 2012--end 2012)}

Development of the infrastructure for new services of the
company. Main requirements were: high availability and
scalability. From April to August, 2012, I've implemented an OAuth v2
specification in an Erlang server (as a part of the SSO solution).

During my experience with the Erlang/OTP language and environment I've
experimented with the open source logging framework and proposed a
pull request with asynchronous mode for this system.

\begin{itemize}[noitemsep, nosep]
  \item \textbf{Languages:} Erlang/OTP
  \item \textbf{Frameworks/Libraries:} Erlang/OTP, Cowboy, Common
Tests, ibrowse, SSL
  \item \textbf{Tools:} Emacs, Bash, Rebar, cURL, Git, Hg
  \item \textbf{Databases:} Mnesia, DETS/ETS
\end{itemize}

\subsubsection*{First project (mid 2007--end 2011}

CRM-CMS system for mobile operators,
\href{https://web.archive.org/web/20150810163328/http://www.billing.ru:80/solutions/customer}{\url{https://web.archive.org/web/20150810163328/http://www.billing.ru:80/solutions/customer}}.

My participation in the project:
\begin{itemize}[noitemsep, nosep]
  \item Porting backend codebase from Windows to Red Hat Enterprise
Linux, HP-UX and Sun Solaris servers
  \item Designing and implementing of a plugins' API for portable
backend system
  \item Refactoring of the Business-Objects layer
  \item Debugging and profiling Unix-related issues of the backend
\end{itemize}

In this project I mostly worked on the backend. Its codebase was
written in C++ and was consisted of Business-Objects layer, Common
Utilities and Business Logic. My first task was to port code base from
Windows platform to Unix-derived ones. I was the only Unix expert in
the team during my work on the project and I've shared expertise with
the team that till present time they are able to support the codebase
on their own. Business-Objects layer consisted of about 400-500
classes and extensively used templating. One day we ran out of IDs in
the link-phase under HP-UX. After that issue my teammate and I have
done the refactoring of the whole Business-Objects layer and reduced
compilation time from 50 min to slightly more than 20 min.

Another task was to design and implement a plugins system for the
Business Logic layer to be able to call pluggable modules in the CRM
workflow.

As I was the only Unix expert in the team, I was working on all the
Unix-related issues of the backend. I've done excessive Valgrind and
GDB runs to find memory leaks, heap misuses, race conditions and
deadlocks as well as to solve global constructors issues of
dynamically loaded libraries.  I've used gprof to periodically find
bottlenecks in libraries. And used to run dtrace for run-time tracing
under Solaris.

During my work on this project I've designed two DSLs (in C++
templates and macros): for embedding XML documents and SQL queries in
C++ code with preprocessor-time checks and compilation,
\href{https://medium.com/@aleksandrvinokurov/dsl-for-xml-in-c-f284453819b}{\url{https://medium.com/@aleksandrvinokurov/dsl-for-xml-in-c-f284453819b}}

I've started automation processes in the team: from automation of the
build and release process to nightly builds and unit-tests.

\begin{itemize}[noitemsep, nosep]
  \item \textbf{Languages:} C++, SQL, PL/SQL, XML, XSLT
  \item \textbf{Frameworks/Libraries:} OTL, Boost Library, Xalan-C,
Xerces-C, Loki, Boost Testing Framework, Google Test Framework, Google
Mocking Framework, Log4cxx
  \item \textbf{Tools:} Emacs, GDB, DTrace, GCC, GNU Autotools, Bash,
Clear Case, Clear Quest
  \item \textbf{Databases:} Oracle 11g
\end{itemize}
